
% whot

\documentclass{article}


\def\dit{\rule[2.5pt]{2pt}{2pt}\hspace{2pt}}
\def\dah{\rule[2.5pt]{6pt}{2pt}\hspace{2pt}}
\def\letterspace{\hspace{4pt}} % in addition to the 2pt added to the end of every dit/dah
\def\wordspace{\hspace{12pt}}

{\renewcommand{\arraystretch}{1.1}
\usepackage{lmodern}

\usepackage{multicol}
\usepackage{indentfirst}

\pagestyle{empty}

\begin{document}
\section*{Threeboard Interface Manual}
    
The Threeboard has three keys, with no labels. 
They are distinguished by their position; left, middle, and right. 
Inputs are done through a series of chords (combinations of 1 or more keys pressed at the same time)
and composed to form the same outputs as a fully-keyed keyboard.

Whenever you see a triplet of 1s and 0s, that represents a chord.
If the manual tells you to press "001 110 111", that means to press the left most key,
then press the right and middle key at the same time, then press all three keys at the same time.

As of the v0 implementation, keystrokes are registered after all keys have been unpressed.
This means that holding down the right key, then holding down the left key, then unpressing the right key
then unpressing the left key, will not send any signals until the left key is unpressed, and it will process as 
101.
This means that each keystroke must be completely separate. 
Due to the lack of visual feedback, it's important that all keystrokes are read perfectly. 

\subparagraph*{}
Morse Code is used as inspiration for the basic alphabet and symbol table,
and allows for some skill transferability. 
There are two morse code input modes, one that uses 3 keys but has no timing requirements,
and a clock morse code mode that requires operators to properly time their keys,
just like a traditional morse code operator. This uses only 1 key. 
This second mode is more difficult and is not yet implemented.

\subsection*{Modes:}
\begin{center}
\begin{tabular} { c|l }
    0 & Morse Code \\
    1 & Modifiers \\
    2 & Arrow Keys \\
    5 & System
\end{tabular}
\end{center}

\subsection*{Global Chords:}
These work regardless of what state or mode the keyboard is in.
\begin{center}
\begin{tabular} { c|l }
    011 & Next mode \\
    111 & Reset to a clean-slate mode 0
\end{tabular}
\end{center}

\subsection*{Morse Code:}
\begin{center}
\begin{tabular} { c|l }
    100 & $\dit$ \\
    010 & $\dah$
\end{tabular}
\end{center}

\subsection*{Morse Code Mode (Mode 0):}
\begin{multicols}{3}

\begin{tabular} { c|l }
     A & \dit\dah \\
     B & \dah\dit\dit\dit \\
     C & \dah\dit\dah\dit \\
     D & \dah\dit\dit \\
     E & \dit \\
     F & \dit\dit\dah\dit \\
     G & \dah\dah\dit \\
     H & \dit\dit\dit\dit \\
     I & \dit\dit \\
     J & \dit\dah\dah\dah \\
     K & \dah\dit\dah \\
     L & \dit\dah\dit\dit \\
     M & \dah\dah \\
     N & \dah\dit \\
     O & \dah\dah\dah \\
     P & \dit\dah\dah\dit \\
     Q & \dah\dah\dit\dah \\
     R & \dit\dah\dit \\
     S & \dit\dit\dit \\
     T & \dah \\
     U & \dit\dit\dah \\
     V & \dit\dit\dit\dah \\
     W & \dit\dah\dah \\
     X & \dah\dit\dit\dah \\
     Y & \dah\dit\dah\dah \\
     Z & \dah\dah\dit\dit \\
\end{tabular} 
    
\columnbreak
\begin{tabular}{ c|l }
     1 & \dit\dah\dah\dah\dah \\
     2 & \dit\dit\dah\dah\dah \\
     3 & \dit\dit\dit\dah\dah \\
     4 & \dit\dit\dit\dit\dah \\
     5 & \dit\dit\dit\dit\dit \\
     6 & \dah\dit\dit\dit\dit \\
     7 & \dah\dah\dit\dit\dit \\
     8 & \dah\dah\dah\dit\dit \\
     9 & \dah\dah\dah\dah\dit \\
    10 & \dah\dah\dah\dah\dah \\
     . & \dit\dah\dit\dah\dit\dah \\
     , & \dah\dah\dit\dit\dah\dah \\
     : & \dah\dah\dah\dit\dit\dit \\
     : & \dit\dit\dah\dah\dit\dit \\
     ' & \dit\dah\dah\dah\dah\dit \\
     - & \dah\dit\dit\dit\dit\dah \\
     / & \dah\dit\dit\dah\dit \\
     ( & \dah\dit\dah\dah\dit \\
     ) & \dah\dit\dah\dah\dit\dah \\
     " & \dit\dah\dit\dit\dah\dit \\
     = & \dah\dit\dit\dit\dah \\
     + & \dah\dit\dah\dit\dah \\
     $\ast$ & \dah\dit\dit\dah \\
     @ & \dit\dah\dah\dit\dah\dit \\
\end{tabular}

\columnbreak
\begin{tabular} { l|l }
    001 & End Letter \\
    001 001 & Space \\
    110 & Reset State
\end{tabular}


\end{multicols}




\subsection*{Modifier Mode (Mode 1):}

After inputting a modifier (001) the keyboard returns to mode 0,
and the next key sent will have the specified modifier. For example, 
\textless C-T\textgreater is 011 100 001 (ctrl) 011 010 001 (shift) 010 001 (t)
\begin{center}
\begin{tabular} { l|l }
    100 \phantom{000}    001 & ctrl   \\
    100 100 001 & win    \\
    010 \phantom{000}    001 & shift  \\
    010 010 001 & alt
\end{tabular}
\end{center}


\subsection*{Arrow Keys Mode (Mode 2):}
You can memorize this easily by imagining 110 as an exemplifier for the key that follows, 
and 110 110 an extension of that.
\begin{center}
\begin{tabular} { l|c|l }
    Code & Keystroke & Move By \\
    \hline
    100 & Left Arrow & Character \\
    001 & Right Arrow & Character \\
    010 & Up Arrow & Line \\
    101 & Down Arrow & Line \\
    
    110 100 & Ctrl Left Arrow & Word \\
    110 001 & Ctrl Right Arrow & Word \\
    110 010 & Ctrl Up Arrow & Paragraph \\
    110 101 & Ctrl Down Arrow & Paragraph \\

    110 110 100 & Home & Entire Line \\
    110 110 001 & End & Entire Line \\
    110 110 010 & Ctrl Home & Entire Document \\
    110 110 101 & Ctrl End & Entire Document \\

    
\end{tabular}
\end{center}    

\subsection*{System Mode (Mode 5):}

\begin{center}
\begin{tabular} { l|l }
    100 & Halt
\end{tabular}
\end{center}




\end{document}

% whot